\documentclass[12pt,a4paper]{article}
\begin{document}
\title{Freemail - Specification and Protocol Documentation}
\date{May 2011}
\maketitle

This is a working draft of the Freemail specification. All parts of this document are subject to
change.

\section{Introduction}
\subsection{What is Freemail}
Freemail is an email-like messaging system that transports all messages over Freenet 0.7 in order to
achieve anonymity and censorship-resilience. Its protocol is designed to be as resistant as possible
to attacks such as message floods and denial of service. Unlike traditional email, it makes it
extremely difficult for others to discover what you have been communicating, who you have been
communicating with, and even that you have been communicating at all.

Freemail uses IMAP and SMTP to interface with standard email clients, taking advantage of interfaces
that people are already accustomed to.

\section{Channel Setup}
\subsection{Mailsites}
Before any communication occurs between a sender and a recipient (who, in accordance with
cryptography tradition shall be called Alice and Bob, respectively), a channel is setup that is used
between those two and only those two parties. This channel comprises a Freenet SSK keypair to which
both Alice and Bob has the private key, permitting two-way communication between those two parties.

All Freemail users have an Freemail address, which one may give out to others in order to allow them
to contact you. From the Freemail address it is possible to derive an identity in the web of trust
that can be used to find the user's 'mailsite'.

A Freemail address comprises an arbitrary text string, followed by an '@' character. Following this
is the base32 encoded representation of the hash of the public key (the first part of an SSK is
this value base64 encoded). The URI must be base 32 encoded in order to make the address case
insensitive to maintain compatibility with traditional email clients. The string '.freemail' is
appended to the whole address. For example, the identity D3MrAR-AVMqKJRjXnpKW2guW9z1mw5GZ9BB15mYVkVc
has the Freemail address <anything>@b5zswai7ybkmvcrfddlz5euw3ifzn5z5m3bzdgpucb26mzqvsflq.freemail

Once the full key for an identity has been obtained the mailpage can be located at
USK@(key of the identity)/Freemail/(edition)/mailpage. The latest edition should be published
through the Web of Trust plugin using the property Freemail.MailpageEdition. The format of a
mailpage is a 'Props File', which is used repeatedly in Freemail as a trivial format for storing
short pieces of information. See section \ref{PropsFile}.

\subsection{Mailpages}
The following pieces of information are required in a mailpage:

\begin{description}
\item[asymkey.modulus] The modulus of the owner's RSA encryption key, as an integer in base 32.
\item[asymkey.pubexponent] The public exponent of the owner's RSA encryption key, as an integer in
	base 32.
\end{description}

\subsection{RTS Messages}
Once Alice has retrieved the recipient's mailpage, she sends an RTS message to Bob. This RTS message
is, again, a props file, with the following keys:

\begin{description}
\item[messagetype] This should be 'rts', to indicate that this message is an RTS.
\item[mailsite] The mailsite key of the sender
\item[recipient] The WoT identity of the recipient
\item[privkey] The private key of the channel
\item[fetchslot] The initial slot the recipient should use for fetching messages
\item[sendslot] The initial slot the recipient should use for sending messages
\end{description}

Following the last data item, there are two carriage-return-line-feeds, followed by Alice's
signature. This is the SHA-256 hash of the message RSA encrypted with Alice's private key, included
as raw bytes.

The final message comprises a randomly generated IV, followed by an AES session key, encrypted with
Bob's public RSA key, followed by this message-signature combination encrypted with this session
key. The encrypted session key must be precicely one RSA block of ciphertext. All bytes after this
are part of the symmetrically encrypted message. The main message is encoded with a block size of
128 bits in CBC mode, with PKCS7 padding.

It is both parties responsibility to keep the private key private. Since only the two parties
involved have the private key, any message that was not inserted by Alice can be assumed to be from
Bob and vice versa.

The 'recipient' field is included to prevent surreptitious forwarding. That is, to prevent Bob from
decrypting the message, leaving Alice's signature intact and encrypting it to someone else (say,
Charlie), who would then be lead in to believing that Alice wished to communicate with him, which is
fact not the case.

This RTS message is then inserted using the Web of Trust plugin to a key derived from the solution
to a CAPTCHA the recipient inserted to Freenet and the sender solved.

Both Alice and Bob then regularly polls the initial slot they use for fetching messages for new
messages or acknowledgements.

\subsection{CTS Messages}
When Bob receives an RTS message from Alice, he decrypts the message using his RSA private key. He
then retrieves the mailsite advertised in the RTS message. Having done this, he reads the signature
on the end and decrypts the signature with the public key he just retrieved from the mailsite. He
then calculates a SHA-256 checksum of the message and checks that his checksum is identical to the
one he has decrypted. If it is not, he must discard the message. This ensures that the message is
really from Alice. He must then read the 'to' field and ensure that its value is identical to his
mailsite URI. If it is not, he must discard the message. This ensures that he is the intended
recipient of the message.

Bob then records the public and private key so that he can poll the key for new messages.

Before doing so, Bob inserts a props file containing messagetype=cts to the first availiable slot.
This completes Bob's part of the channel setup procedure.

Alice should check periodically for the insertion of this CTS message. If it does not arrive, Alice
should re-send the RTS message. The client may try several times before declaring the message
undeliverable.

\section{Message Exchange}
\subsection{The Messages}
Messages are inserted to a key derived from the 'privkey' field and one of the initial slots. The
party that sends the RTS, the initiator, uses the fetchslot for inserting messages and the sendslot
for receiving messages, which the recipient uses the slots for the opposite purpose. The full key
used for inserting messages is then SSK@$<$privkey$>$/$<$role$>$-$<$slot$>$, where role is i for the initiator
and r for the recipient. The slot is initially the value indicated in the RTS. Each subsequent
slot is the SHA-256 digest of the previous slot, forming a seqence of message slots. This hash
sequence gives forward security, provided that clients destroy values of the slots, and the inital
slot once they have been used. Alice should insert a new message to the first slot to which
inserting does not causes a collision. Formulaically, Bob first polls the key:

SSK@$<$SSK key base$>$/i-$<$fetchslot$>$

Once Bob has successfully retrieved this key, he begins to periodically request the key:

SSK@$<$SSK key base$>$/i-$<$H(fetchslot)$>$

It is recommended that clients poll several messages ahead rather than just the immediately next
message, since simulations suggest that it is possible for single keys not to be retrievable in this
kind of circumstance. So for example, once Bob has sent his CTS messages, he should start polling
for the keys: \\
\\
SSK@9GXtGxN4CEJD~8a\-307V6yzyhl8Gx5U\-YbWVDTEyUXH6o,gDWfr2CqVm\-
DAeJurKF2iieM\-5AkjXstOl2V5j\-yuTHeo4,AQABAAE/i-$<$fetch slot$>$ \\
\\
SSK@9GXtGxN4CEJD~8a\-307V6yzyhl8Gx5U\-YbWVDTEyUXH6o,gDWfr2CqVm\-
DAeJurKF2iieM\-5AkjXstOl2V5j\-yuTHeo4,AQABAAE/i-$<$H(fetch slot)$>$ \\
\\
and \\
SSK@9GXtGxN4CEJD~8a\-307V6yzyhl8Gx5U\-YbWVDTEyUXH6o,gDWfr2CqVm\-
DAeJurKF2iieM\-5AkjXstOl2V5j\-yuTHeo4,AQABAAE/i-$<$H(H(fetch slot))$>$

Alice can insert a message to the lowest numbered key of this pattern that does not cause a
collision whenever she chooses. These comprise a number of properties (in the same way as a props
file) followed by a double carriage-return-line-feed. Following this is the standard MIME mail
messages. No signing or encryption is used here, since at this stage it is achieved inside Freenet
by virtue of only Alice and Bob knowing the SSK upon which they communicate.

The mandatory properties of the props file are 'id' and 'messagetype'. 'id' is a string that must
uniquely identify the message from any other past or future message transported using the same
channel. Bob must check this value and ensure that he has not already received this message id. If
he has, he discards the message, but still acknowledges his receipt of it. All clients must ignore
any unknown keys and begin reading the message only at the double line break in order that extra
properties can be added in the future.

\subsection{Message Acknowledgements}
When Bobs client receives a message, it reads it and passes it onto Bob. He then
inserts a propsfile to the next availiable slot with 'messagetype' set to ack, and 'id'
set to the value of 'id' from the message in question. For example, if Bob has just fetched a message
from key: \\
\\
SSK@9GXtGxN4CEJD~8a\-307V6yzyhl8Gx5U\-YbWVDTEyUXH6o,gDWfr2CqVm\-
DAeJurKF2iieM\-5AkjXstOl2V5j\-yuTHeo4,AQABAAE/$<$fetch slot$>$ \\
\\
With the contents: \\
\\

\fbox{\begin{minipage}[h]{400pt}
id=657488664753 \\

To: Bob Burton $<$bob@longkey.freemail$>$ \\
From: Alice Andrews $<$alice@anotherlongkey.freemail$>$ \\
Subject: Eve \\

I think Eve from down the road might be trying to spy on us. I've never liked the looked of her, you
know. It's always the quiet ones.
\end{minipage}} \\
\\
Then he might insert an acknowledgement to the key: \\
\\
\\
SSK@AJoZbUvGkXlAJwI\-jdbu9BLPhpIXBu6\-w6nGwKYBnMfNLi,ACEgE1uUIzJdC\-
Xcsz1yjgW45u\-Az-KuMrXBFYG\-U8maqc/$<$send slot$>$ \\

With the content:\\

\fbox{\begin{minipage}[h]{400pt}
messagetype=ack\\
id=657488664753
\end{minipage}} \\

If Alice has, for whatever reason, not received a
CTS message from Bob, her receipt of a message ack should additionally be treated as receipt of a
CTS message.

\appendix

\section{Props Files}
\label{PropsFile}
A props file is a sequence of keys and values. Keys and values are separated by a single equals sign
('=') and lines are separated by a carriage return and line feed ($\backslash$r$\backslash$n), with
the exception that if the propsfile will only be read locally, it is permissable to use the line
separator native to the local machine. For example, for props files that are never transmitted over
the network, it is permissable to use just a line feed ($\backslash$n) to separate lines. It is
recommended for simplicity, though not required, that the keys be lowercase and contain only
alphanumeric characters. The keys must not contain the equals sign, as there is no mechanism for
escaping equals signs. The value may contain equals signs and therefore parsers of this format must
treat and equals signs after the first on any line as part of the value text.

An example of a propsfile is below: \\
\\
\fbox{\begin{minipage}{400pt}
name=Bob Burton \\
age=39 \\
occupation=Builder \\
Pet's Name=Stevie the Sycophantic Squirrel \\
\end{minipage}}

\end{document}
